% vim: shiftwidth=2:tabstop=2:softtabstop=2:expandtab:textwidth=120:colorcolumn=120
\documentclass{llncs}
%\documentclass[llncs]{IEEEtran}

\usepackage{microtype} %This gives MUCH better PDF results!
%\usepackage[active]{srcltx} %DVI search
\usepackage[cmex10]{amsmath}
\usepackage{amssymb}
\usepackage{fnbreak} %warn for split footnotes
\usepackage{url}
%\usepackage{qtree} %for drawing trees
%\usepackage{fancybox} % if we need rounded corners
%\usepackage{pict2e} % large circles can be drawn
%\usepackage{courier} %for using courier in texttt{}
%\usepackage{nth} %allows to \nth{4} to make 1st 2nd, etc.
%\usepackage{subfigure} %allows to have side-by-side figures
%\usepackage{booktabs} %nice tables
%\usepackage{multirow} %allow multiple cells with rows in tabular
\usepackage[utf8]{inputenc} % allows to write Faugere correctly
\usepackage[bookmarks=true, citecolor=black, linkcolor=black, colorlinks=true]{hyperref}
\hypersetup{
pdfauthor = {Vijay Ganesh, Trevor Hansen, Dan Liew, Ryan Govostes, Khoo Yit Phang, Mate Soos},
pdftitle = {STP},
pdfsubject = {SMT Competition 2014},
pdfkeywords = {SMT Solver},
pdfcreator = {PdfLaTeX with hyperref package},
pdfproducer = {PdfLaTex}}
%\usepackage{butterma}

%\usepackage{pstricks}
%\usepackage{graphicx,epsfig,xcolor}
%\usepackage[algoruled, linesnumbered, lined]{algorithm2e} %algorithms
\setlength{\parskip}{1ex}
\begin{document}
\title{STP}
\author{Vijay Ganesh, Trevor Hansen, Dan Liew, Ryan Govostes, Khoo Yit Phang, Mate Soos}
\institute{}

\maketitle
\thispagestyle{empty}
\pagestyle{empty}

\section{Introduction}
STP\cite{Vijay:Thesis:2007} is an efficient open source solver for QF\_BV and arrays without extensionality. STP
recursively simplifies bit-vector constraints, solves linear bit-vector equations, and then eagerly encodes them to CNF
for solving. Array axioms are added as needed during an abstraction-refinement phase.

The version of STP submitted to STMCOMP 2014 is revision fcfb30e8664 of STP's publicly available source code
repository~\cite{STP:github}. It was compiled with a slightly tuned version of
CryptoMiniSat~\cite{DBLP:conf/sat/SoosNC09} revision 7ae6c5123 available from its own public
repository~\cite{CMS:github}.


\section{Development history}
STP was originally developed by Vijay Ganesh under the supervision of Professor David Dill. Later releases were
developed by Trevor Hansen under the supervision of Peter Schachte and Harald Søndergaard. STP handles arbitrary
precision integers using Steffen Beyer's library. STP encodes into CNF via the and-inverter graph package ABC of Alan
Mishchenko~\cite{Brayton:2010:AAI:2144310.2144317}.

We found many defects using Robert Brummayer and Armin Biere's
fuzzing and delta debugging tools~\cite{Brummayer:2009:FDS:1670412.1670413} in both STP and CryptoMiniSat. Although
these tools are not a proper replacement for unit or integration testing, they provide a healthy sanity check against
some forms of bugs. In particular, fuzzing does not test against performance regressions,
instead it can only detect where certain optimizations are wrongly implemented such as to produce incorrect results.

In the past year, STP has been actively developed on \href{https://github.com/stp/stp}{GitHub}.

\section*{Acknowledgements}
Thanks for everyone who submitted bug reports and pull requests and other useful data such as test cases.

\bibliographystyle{splncs}
\bibliography{sigproc}

\vfill
\pagebreak

\end{document}
