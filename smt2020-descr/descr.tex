% vim: shiftwidth=2:tabstop=2:softtabstop=2:expandtab:textwidth=120:colorcolumn=120
% \documentclass{llncs}
\documentclass{easychair}
%\documentclass[llncs]{IEEEtran}

\usepackage{microtype} %This gives MUCH better PDF results!
%\usepackage[active]{srcltx} %DVI search
% \usepackage[cmex10]{amsmath}
% \usepackage{amssymb}
\usepackage{fnbreak} %warn for split footnotes
\usepackage{url}
%\usepackage{qtree} %for drawing trees
%\usepackage{fancybox} % if we need rounded corners
%\usepackage{pict2e} % large circles can be drawn
%\usepackage{courier} %for using courier in texttt{}
%\usepackage{nth} %allows to \nth{4} to make 1st 2nd, etc.
%\usepackage{subfigure} %allows to have side-by-side figures
%\usepackage{booktabs} %nice tables
%\usepackage{multirow} %allow multiple cells with rows in tabular
\usepackage[utf8]{inputenc} % allows to write Faugere correctly
% \usepackage[bookmarks=true, citecolor=black, linkcolor=black, colorlinks=true]{hyperref}
% \hypersetup{
% pdfauthor ={STP},
% pdftitle = {STP},
% pdfsubject = {SMT Competition 2020},
% pdfkeywords = {SMT Solver},
% pdfcreator = {PdfLaTeX with hyperref package},
% pdfproducer = {PdfLaTex}}
%\usepackage{butterma}

%\usepackage{pstricks}
%\usepackage{graphicx,epsfig,xcolor}
%\usepackage[algoruled, linesnumbered, lined]{algorithm2e} %algorithms
\setlength{\parskip}{1ex}
\begin{document}
\title{STP in the SMTCOMP 2020}
\author{Various}
\institute{}

\maketitle
\thispagestyle{empty}
\pagestyle{empty}

\section{Introduction}
STP\cite{Vijay:Thesis:2007} is an open-source solver for QF\_BV and arrays without extensionality. 
STP recursively simplifies bit-vector constraints, solves linear bit-vector equations, and then eagerly encodes them to CNF for solving. 
Array axioms are added as needed during an abstraction-refinement phase.

\section{Development history}
STP was originally developed by Vijay Ganesh under the supervision of Professor David Dill. 
Later releases were developed by Trevor Hansen under the supervision of Peter Schachte and Harald Søndergaard. 
STP handles arbitrary precision integers using Steffen Beyer's library. 
STP encodes into CNF via the and-inverter graph package ABC of Alan Mishchenko~\cite{Brayton:2010:AAI:2144310.2144317}.
STP supports different SAT backends: MiniSat~\cite{MiniSat:github}, CryptoMiniSat~\cite{CMS:github} and Riss~\cite{Riss:github}.

\section{Recent Developments to STP}
In the last year:

Andrew V. Jones, Mate Soos, Christian Cadar, Brian Foley, Ryan Govostes, Gleb Popov, and Ondřej Súkup have improved build system. 

Trevor Hansen has improved how speculative simplifications are applied.


\section{Recent Developments to the Underlying SAT Solvers}
CryptoMinisat (\emph{CMS}) has been augmented with the Stochastic Local Search (SLS)~\cite{DBLP:conf/sat/CaiLS15} solver CCAnr. 
CryptoMiniSat aims to be a modern, open-source SAT solver using inprocessing techniques, optimized data structures and finely-tuned timeouts to have good control over both memory and time usage of inprocessing steps. It also supports, when compiled as such, to recover XOR constraints and perform Gauss-Jordan elimination on them at every decision level. CryptoMiniSat is authored by Mate Soos. CCAnr~\cite{DBLP:conf/sat/CaiLS15} is a stochastic local search (SLS) solver for SAT, which is based on the configuration checking strategy and has good performance on non-random SAT instances. 

The SAT solver MergeSAT is a fork of the winner of the SAT Competition 2018, MapleLCMDistChronoBT. MergeSAT is authored by Norbert Manthey. 
MergeSAT ports features that have been proposed in other solvers into that solver, enables incremental solving again, as well as makes the solver behavior reproducible again.

\section*{Acknowledgements}
Vijay Ganesh, Dan Liew and Ryan Govostes contributed substantialy to the STP code base.

\bibliographystyle{splncs}
\bibliography{sigproc}

\vfill
\pagebreak

\end{document}

